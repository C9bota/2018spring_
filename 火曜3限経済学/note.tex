\documentclass{jsarticle}
\begin{document}
\title{経済学}
\author{久保田知恵}
\maketitle

\section{貨幣価値}
\subsection{円高・円安ってなんだ?}
「円高・円安」などと騒がれてきたのは1970年ごろから。
1971年までは固定性だった(1ドル360円)
1971年からは貨幣変動制となった。
ドルなど他国の貨幣との対比で円高円安について語られることが
多いため、円高・円安とは外貨が絡むものだと言う先入観が
植えつけられている。\\
\subparagraph{問題の回答}
問題:日本国内の物価が上昇すると、円の価値が下落する。
円安というのは、円の価値が下落するということである。
故に、日本国内の物価が上昇すると円安になる。
以上の推論は正しいか。\\
それぞれの文は正しい。一般に、日本国内の物価が上昇すると、
円安圧力となる。\\
円の価値:日本国内でものを買うために必要な貨幣の量が必要に
なる
①円の値段:商品に対する円の購買力・国内市場における円の商品購買力\\
②円の値段:外貨に対する円の値段・外国為替市場における\\
これを踏まえると、\\
日本国内の物価が上昇すると、
円の商品購買力が下落する。
円安というのは、他の通貨の購買力が下落するということである。\\



?外貨と他通貨って何が違うのか\\
例えば、日本で主に取引される仮想通貨は他通貨と言えるのか。
「通貨が売り買いされる「マーケット」は外国為替市場だけ」






\subsection{需要法則}
例題:ペン1本が日本の市場では1ドルで販売されていた。為替相場
は1ドル=50円であった。
\begin{enumerate}
  \item このとき、為替相場にはドル高・ドル安どちらの
  圧力が作用するか。誰のどのような取引行為によって
  そのような圧力が作用するのか。というそのプロセスも
  示しなさい。
  \item この相場変動圧力は「1ドル=〜円」になれば消滅するのか。
\end{enumerate}
回答:
\begin{enumerate}
  \item 日本の需要者による対米輸入増大(ドル建て)がおこり、これにより、
  円売りドル買いが増大する。その結果、円安ドル高の圧力
  がかかる。
  \item 1ドル=100円になれば、日本で購入しても米国から購入しても
  1本のペンの値段が無差別になるため。

\end{enumerate}

\begin{itemize}
  \item 実需の動きに基づく通過売買の増減(=レート変動圧力)
  を発生させない為替相場基準\\
  実需とは、この場合「ペンを使うため」ということになる。実需の
  反対の言葉としては、投機的需要という言葉が当てられる。
  \item 各国内に置ける通過の商品購買力を反映した為替レート。
  \item レート変動の重心:ここからレートがずれるとここに引き戻す圧力が
  発生する。
\end{itemize}
同じ
ドル買いという行為に対して、買ったドルでペンを買うのか、
為替により儲けたいから買うのかという違いが出てくる。後者は
、値上がり益を求めての需要と言える。
\\参考テキスト:59ページ
為替のうち、1\%は実需、その他99\%が投機的実需によるものである。\\
グラフの見方\\
長期的に見ると、赤い線を中心として実際の円相場は上下している。\\

\subsection{確認問題}
日本の景気がいいと、海外から日本に投資資金が流入する。その逆も然り。
しかし、日本国内の資金が増減することはない。\\
投資家Aは日本の景気がよくないため、資金を日本から
アメリカに移す。その時Aは円売りドル買を行う。\\
具体例:
\begin{enumerate}
  \item Aによる円売りドル買 1ドル=100円の時、100万円売り、1万ドルを買う。
  \item C銀行(日本支店では円預金、米国支店ではドル預金を扱う)
  \item 日本:Aの口座からC銀行の預金へ100万円振り込まれる
  \item 米国:C銀行の預金から、Aの口座へ1万ドル振り込まれる

\end{enumerate}
Aの口座からC銀行に振り込まれただけなので、日本国内でのお金の量は変わっていない。
米国国内のお金に関しても同様。

\section{円売ドル買の具体例}


\paragraph{貨幣価値という言葉の意味}
\begin{itemize}
  \item 外国為替市場 \\
  貨幣の価格
  \item 国内の財市場
  \item 貨幣市場
  \item その他\\
  希少性(希少価値)


\end{itemize}

\paragraph{問題1-3}
貨幣価値が下落すると物価が上昇するのか、それとも物価が上昇すると貨幣価値が下落するのか。\\
\subparagraph{回答1-3}
日本政府が国民にお金を配布すると、国民の手元のお金が増える。
すると、お金の価値(希少性)が下がり、気軽にお金を使えるようになる。
需要が増大するため、物価は上昇する。お金の価値(商品購買力])は下がる。

\paragraph{例題03}
1ドル=200円で合った。この時円相場(対ドル)
が変動し、その変動率は-20で」あった。
\begin{enumerate}
  \item 変動の結果1ドル=何円になったのか。
  250円になった。
  \item この時のドル相場(対円)の変動率を求めよ。
  25パーセント
\end{enumerate}

\section{需要と供給}
\subsection{需要と供給の定義}
\paragraph{例題2}
市場の鉛筆の供給量が1本であるとき、この鉛筆
を欲しいと思っている人が2人いる場合、需用量と
供給量とではどちらが多いと言うことになりますか。\\

経済活動は、生産活動・分配活動・消費活動
に分けられる。
経済学の用語としては、需要とはただ単に欲しい・必要
だという状況ではない。購買力を含め需要と言うことができる。
反対に、供給とはただ与えることでなく、価格が設定された上で市場に出されている
状態を供給と言う。購買力・価格が示されていないので、例題2
からは需要と供給にどのような関係があるかを断定することは
できない。\\
需要=購買力を伴う欲望 供給=一定の価格の下での提供

\paragraph{例題2ー改}
市場の鉛筆の供給量が1本であるとき、この鉛筆
を欲しいと思っている人が2人いる場合、需用量と
供給量とではどちらが多いと言うことになりますか。
ただし、鉛筆1本の値段は100円で、Aの所持金は100円、
Bの所持金は50円とする。

\paragraph{例題2ー回答}
Aは需要にカウントされ、Bは需要にカウントされるため、
需用量==供給量となる。このことを、需給一致状態、
市場均衡状態、安定均衡状態であるという。特に、安定
均衡状態は、価格を動かす力が市場には存在しない状態
のことを言う。
この時の鉛筆1本は均衡取引量で、100円は均衡価格である。
\subparagraph{補足}
鉛筆1本の値段が80円の場合、均衡価格が80円になるだけで、
需給一致状態であることには変わらない。\\

 限界費用逓増=収穫逓減

\subsection{需要法則を考えよう}
\begin{table}[]
\centering
\caption{My caption}
\label{my-label}
\begin{tabular}{l|lll}
価格  & 150円 & 130円 & 100円 \\ \hline
本数  & 1本   & 2本   & 3本   \\
収入  & 150円 & 260円 & 300円 \\
コスト & 80円  & 180円 & 300円 \\
利潤  & 70円  & 80円  & 0円
\end{tabular}
\end{table}

価格が上昇したことにより、供給量が増大し、需要量が減少した。
Cは市場から排除され、需要は一致となった。


\subsection{社会的余剰}
A=150円 B=130円 C=100円
社会的余剰=生産者余剰(利潤)+消費者余剰(消費者のお得度)
\subparagraph{例}
値段が150円だったときの生産者余剰と消費者余剰はどうなるかを考える。
生産者余剰が70円、消費者余剰は0円なので、社会的余剰は70円。\\
値段が130円だったときの生産者余剰は80円、
消費者余剰は20円なので、社会的余剰は100円。
値段が100円だったときの生産者余剰は0円
、消費者余剰は80円なので、社会的余剰は80円。\\
つまり、値段が130円のときにこの場合最も社会的余剰が大きくなる。
ミクロ経済学において自由競争下では、市場均衡点において社会的余剰は最大となる。
p18,19をかくにんすること

\subsection{需要法則まとめ}
需要が増大し、供給量より大きくなり価格上昇圧力がかかる。
そして、需要量が減少&供給量の増大がおこる。
需給一致点に達し、価格上昇は停止する。 \\
需要量増減は需要増減とは違う。供給量増減と供給増減も同様に。
__量は、価格変動の結果を表し、__増減は価格変動の原因となっているもの
と定義されている。



\section{需給法則}
\subsection{需要曲線}
脱落ということではないが、需要曲線は左に移動する。
\subsection{供給曲線}


\end{document}
