\documentclass{jsarticle}
\begin{document}

\title{多変量解析}

\author{久保田知恵}
\maketitle

\section{第1回}
\section{第2回:回帰分析1<単回帰と最少2乗法>}
\subsection{回帰モデル}


\subsection{線形解析モデル}
現実的に考えれば線形で考えても問題ないのではないかというモデル。
例:日本人の身長と体重\\
説明変数が1個だけの場合を線形単回帰モデル、複数個の場合を重回帰モデル
という。
\subsection{線形単回帰モデル}
Y=arufa + be-taX
(X,Y)の観測値
(x1,y1),(x2,y2),......,(xn,yn)
例:熊谷の1日の最高気温と最低気温一年分を調べてみよう\\
変数1個だけで表して見たい。
\subsection{道パラメータの推定}
y1=alfa+betax1 と考えて行きたいが、観測値には誤差があるので
epsoronを「残差」として導入してあげる。ハットがついたものは推定値
ということは、epsion1=y1-Yハットが成り立つ。
\subsection{残差に関する仮定}
統計には仮定がつきもの。この先の議論を進めるために、残差についていくつかの
仮定を施す。\\
\begin{itemize}
  \item 残差の期待値は0であることを認める。
  誤差はマイナスにもプラスにも触れるということは、その平均を見た場合、
  その値は0になるであろうという考え。もし、0でなかったら、その定数分
  をalfaに足してあげれば良い。
  \item 残差の出方(分散)は一定(定数)であってほしい。
  \item 相異なる個体の残差は無相関である。
  誤差同士には何も相関はないよね、という考え方。それよりも今はYとXの
  相関について調べている。
\end{itemize}
\subsection{残差の評価尺度}
残差の評価尺度には次の3つがあげられる。\\
残差:residual
\begin{itemize}
  \item 残差平方和
  残差の二乗和
  \item 絶対残差和
  1つ1つ
  \item 最大絶対残差
  一番大きな残差が出ているところが小さくなるように働きかける。\\
\end{itemize}

\subsection{最少2乗法}
なぜ、RSSがよく使われるのか。凸関数であるため、極小値がひとつだけである。
もう1つは、2次導関数が常に正である。\\
課題:RSSをalfa,bataどちらについても偏微分に行い形を整える。
求めるべきalfa,bataに*をつけ、連立方程式として解く。

\subsection{回帰分析の語源}
並外れた値を集めて、もう一度サンプルをとると、平均に回帰する。一般的に、回帰分析
の対象は相関が極めて強いデータに対して用いられる。

\section{第3回:}
予習:重回帰モデルについて、ベクトル・行列を用いて連立式を解けるように。

\section{第4回}
\section{第5回}
\subsection{主成分分析とは}
\paragraph{2変数での例}
\subsubsection{軸の回転}
\subsubsection{主成分を求める基準}
\subsubsection{ラグランジュの未定乗数法}
\subsection{固有値問題}
\subsection{第2主成分}
\subsection{変量の標準化}

\section{第6回}
\subsection{主成分分析(N変量)の解法}
2変量での解法は___である。
これをN変量分まで拡張する。すなわち、行列をN次にすればよい。
ここでN次の固有方程式の回を求めるという話が出てくるが、5次以上は代数
的に解を求めるための一般的な方法がないためできない。
そこでどうするか、数値計算で求めよう

\paragraph{べき乗法}
イプシロンは10の−7乗
問題点:絶対値最大の固有値が重解だと適用できない
他の固有値を出したい場合はヴィーラント減次を使う

\paragraph{QR法}
QR分解を何回も繰り返してあげることで三角行列を作ってやる。
Q直行行列
←は代入、-> は変換を示す
下三角成分が十分小さくなったときに、斜めのところが固有値になるよ
固有ベクトルではない

\subsection{寄与率}
どのくらいで表現すべきかの尺度
全部の固有値がわからないと出ない・・?
分散共分散行列

\subsection{主成分得点}
?分散共分散行列
?ノルム
得られた主成分得点ってなんだ
第二主成分は第1に対して直交している
2次元平面にプロットしたらわかりやすくなるんじゃ・・・?

\subsection{負荷量}
変量がどれだけ主成分に影響をあたえるかの指標

因子分析は中間後に出てくる

\section{第7回:因子分析}
\subsection{潜在的因子}
\subsection{因子分析のモデル}
\subsection{因子分析の応用分野}
アンケートの裏にある望み(潜在的因子)を評価したいとき、特に人間が関わるマーケティング
の分野で応用されることが多い。
\subsection{主成分分析との対比}
\subparagraph{主成分分析}
観測可能な大量な変量を要約、いかに少なくして表現するかが大切。
\subparagraph{因子分析}
観測可能な変量を結果として、その結果を引き起こした原因を探ることが目的。
\subsection{行列表現}
とりあえず大量のデータが出てきたら、行列
\subsection{直交解}
\subparagraph{仮定1:}Fi,Ukの平均は0
\subparagraph{仮定2:}Fi,Ukは互いに無相関
?共分散行列
この過程を満たすFを直交解という。
11ぺーじから




















\end{document}
