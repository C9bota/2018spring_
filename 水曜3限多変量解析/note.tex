\documentclass{jsarticle}
\begin{document}

\title{多変量解析}

\author{久保田知恵}
\maketitle

\section{第1回}
\section{第2回:回帰分析1<単回帰と最少2乗法>}
\subsection{回帰モデル}


\subsection{線形解析モデル}
現実的に考えれば線形で考えても問題ないのではないかというモデル。
例:日本人の身長と体重\\
説明変数が1個だけの場合を線形単回帰モデル、複数個の場合を重回帰モデル
という。
\subsection{線形単回帰モデル}
Y=arufa + be-taX
(X,Y)の観測値
(x1,y1),(x2,y2),......,(xn,yn)
例:熊谷の1日の最高気温と最低気温一年分を調べてみよう\\
変数1個だけで表して見たい。
\subsection{道パラメータの推定}
y1=alfa+betax1 と考えて行きたいが、観測値には誤差があるので
epsoronを「残差」として導入してあげる。ハットがついたものは推定値
ということは、epsion1=y1-Yハットが成り立つ。
\subsection{残差に関する仮定}
統計には仮定がつきもの。この先の議論を進めるために、残差についていくつかの
仮定を施す。\\
\begin{itemize}
  \item 残差の期待値は0であることを認める。
  誤差はマイナスにもプラスにも触れるということは、その平均を見た場合、
  その値は0になるであろうという考え。もし、0でなかったら、その定数分
  をalfaに足してあげれば良い。
  \item 残差の出方(分散)は一定(定数)であってほしい。
  \item 相異なる個体の残差は無相関である。
  誤差同士には何も相関はないよね、という考え方。それよりも今はYとXの
  相関について調べている。
\end{itemize}
\subsection{残差の評価尺度}
残差の評価尺度には次の3つがあげられる。\\
残差:residual
\begin{itemize}
  \item 残差平方和
  残差の二乗和
  \item 絶対残差和
  1つ1つ
  \item 最大絶対残差
  一番大きな残差が出ているところが小さくなるように働きかける。\\
\end{itemize}

\subsection{最少2乗法}
なぜ、RSSがよく使われるのか。凸関数であるため、極小値がひとつだけである。
もう1つは、2次導関数が常に正である。\\
課題:RSSをalfa,bataどちらについても偏微分に行い形を整える。
求めるべきalfa,bataに*をつけ、連立方程式として解く。

\subsection{回帰分析の語源}
並外れた値を集めて、もう一度サンプルをとると、平均に回帰する。一般的に、回帰分析
の対象は相関が極めて強いデータに対して用いられる。

\section{第3回:}
予習:重回帰モデルについて、ベクトル・行列を用いて連立式を解けるように。












\end{document}
