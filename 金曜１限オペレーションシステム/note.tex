\documentclass{jsarticle}
\begin{document}

\title{オペレーションシステム}

\author{久保田知恵}
\maketitle

\section{第1回}
特定のOSに注目するものではない
資源:計算を行うのに利用できる存在
公平性と効率性のバランス
API:
ABI:
仮装計算機
資源(メモリ、ディスク)の違いの差を全て意識してつくらなくてもよい
?コンテナとVMのメリットの違い

UNIX
Multics→マルチ → ユニ
基本的な考えは同じ
ベル研究所(C言語開発したところ)

Apple macOS
もともとUNIXとは無関係
→macOSXの時に設計を刷新
\section{第2回}
\section{第3回}
\section{第4回}
\subsection{ジョブの性質}
\subsection{スケジューリング}
\subsection{ポリシーの比較基準}
\begin{itemize}
  \item 公平かどうか
  \item 単純かどうか
  \item CPUの利用率
  \item スループット
\end{itemize}
このほかに、レスポンスタイムとターンアラウンドタイムが比較基準となる。
\subparagraph{レスポンスタイム}
\subparagraph{ターンアラウンドタイム}
バッチジョブの場合はこちらを重視\\
それぞれどう言う尺度を求められている。\\
\begin{itemize}
  \item 先着順
  \item 最短時間順
  \item 後着順
  \item 優先度順
  \item ランダム(抽選型)
  \item ラウンドロビン
  \item 複合型
  ほとんどはこれ。ラウンドロビン+何か
  \item 経過処理時間順
  \item 残余処理時間順
\end{itemize}
優先順位の高いプロセスに対して、いつディスパッチすべきか。今のOSはプリエンティブ・
・・仕事の途中であってもディスパッチ?
される。
ノンプリエンプティブ・・・実行中のプロセスがCPUを手放さないとディスパッチされない
\subparagraph{先着順}
First-Come,First-Served:FCFSともよばれる。FIFOとも。
とてもシンプル。先に並んだ方が有利なので、ディスパッチはない。
例:バッチジョブ。デメリットは、時間のかかる仕事が先に並ぶとキューが伸びる。
図_の例では、ターンアラウンドタイムはA=4,B=9,C=11(単位時間)であり、
平均は8時間である。
\subparagraph{最短時間順}
Shortest Job First:SJF 短い仕事ほど優先順位
が高い。平均待ち時間は(理論上)最短になる。しかし、実際には
利用ができない。なぜなら、同じアルゴリズムでもプロセス終了
にかかる時間が異なることがある(むしろ多い)からである。
理論値であるからこそ、ベンチマークの指標として使われる
ことが多い。ターンアラウンドタイムはA=7,B=12,C=3
で、平均は7.3単位時間である。
\subparagraph{優先度順}
(fixed:固定された) priority scheduling あらかじめ優先度
を割り振り、その優先度が優先順位となる。固定でない(
プロセス途中で優先度が上下する)場合もある。
\subparagraph{ラウンドロビン}
総当たり戦、円卓会議のこと。優先順位はなく、公平。
そのかわり、一定時間(=タイムスライス)ごとにプロセスを切り替える
タイムスライスを長くすれば、FCFSに、短くすれば、
最短順に近く。あまりタイムスライスを短くすると、プロセス
を切り替える際のオーバーヘッドが増加してしまう。
\subparagraph{その他}
\begin{itemize}
  \item 経過処理時間順
  shortest elapsed time first CPUの利用時間を累積
  させ、経過時間の短いものを優先。公平性がある。
  利用時間を保持しなければいけない。
  \item 残余処理時間順
  shortest remaing time first 残り時間が短いものから優先。
  残りがどのくらいかわからないと言う問題
  \item 降着順
\end{itemize}
\subsection{CPU待ち時間}
待ち時間が長ければ長いほど優先度を上げてあげよう。→公平性
resouce starvation CPU利用時間に応じて優先度を下げよう。\\
実装例:多段フィードバックキュー
\subparagraph{多段フィードバックキュー}
nice値を用いて、優先度を設定(windowsは32段階)。基本はラウンドロビン・先着
順で処理する。CPU利用時間・待ち時間で優先度を上下してやる
\subsection{CFS}
Completely Fair Scheduling スッゲー完璧なスケジューリング。
二分探索木を作りながら、仮装経過時間のみで決定する。

\section{第5回}



\end{document}
