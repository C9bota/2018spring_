\documentclass{jsarticle}
\begin{document}
\title{経済学}
\author{久保田知恵}
\maketitle

\section{貨幣価値}
\subsection{円高・円安ってなんだ?}
「円高・円安」などと騒がれてきたのは1970年ごろから。
1971年までは固定性だった(1ドル360円)
1971年からは貨幣変動制となった。
ドルなど他国の貨幣との対比で円高円安について語られることが
多いため、円高・円安とは外貨が絡むものだと言う先入観が
植えつけられている。\\
\subparagraph{問題の回答}
問題:日本国内の物価が上昇すると、円の価値が下落する。
円安というのは、円の価値が下落するということである。
故に、日本国内の物価が上昇すると円安になる。
以上の推論は正しいか。\\
それぞれの文は正しい。一般に、日本国内の物価が上昇すると、
円安圧力となる。\\
円の価値:日本国内でものを買うために必要な貨幣の量が必要に
なる
①円の値段:商品に対する円の購買力・国内市場における円の商品購買力\\
②円の値段:外貨に対する円の値段・外国為替市場における\\
これを踏まえると、\\
日本国内の物価が上昇すると、
円の商品購買力が下落する。
円安というのは、他の通貨の購買力が下落するということである。\\



?外貨と他通貨って何が違うのか\\
例えば、日本で主に取引される仮想通貨は他通貨と言えるのか。
「通貨が売り買いされる「マーケット」は外国為替市場だけ」






\section{}
\section{}
\section{}
\section{}
\section{}
\section{}
\section{}
\section{}
\section{}
\section{}
\section{}


\end{document}
